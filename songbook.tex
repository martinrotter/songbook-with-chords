\documentclass[12pt,oneside]{article}
\usepackage[top=0.5in,bottom=0.7in,right=0.5in,left=0.5in,headheight=65pt]{geometry}
\usepackage[usenames,dvipsnames]{xcolor}
\usepackage{graphicx}
\usepackage{fancyhdr}
\usepackage{array}
\usepackage{leadsheets}

% Setup leadsheets.
\useleadsheetslibrary{musejazz}
%\useleadsheetslibraries{musicsymbols}

% Custom title template.
\definesongtitletemplate{title-templ}{
\ifsongmeasuring
{\section*}
{\section}{%
\songproperty{title}%
\ifsongproperty{music}
{ \small\textsuperscript{by \songproperty{music}}}
{}%
}}

\setleadsheets{
  title-template = title-templ,
  align-chords = l,
  bar-shortcuts = true,
  verse/numbered = true,
  chords/format = \color{blue}\sffamily\large\bfseries
}

% Print piano chord diagrams.
\newcommand{\chordDiagram}[1]{\raisebox{13pt}{\writechord{#1}} \includegraphics[width=5cm]{Chords/#1.png}}

% Add footer.
\pagestyle{fancy}
\fancyhead{}
\renewcommand{\headrulewidth}{0pt} % no line in header area
\fancyfoot{} % clear all footer fields
\fancyfoot[R]{\thepage}           % page number in "outer" position of footer line
\fancyfoot[L]{\small Only for non-commercial and educational purposes. Chord diagrams are taken from https://www.pianochord.org.} 

% No numbering for \section.
\setcounter{secnumdepth}{0}

\begin{document}
\tableofcontents
\newpage

\begin{song}{title={Falling Slowly}, music={Glen Hansard \& Markéta Irglová}}
\begin{center}
\begin{tabular}{rr}
\chordDiagram{C} & \chordDiagram{F} \\
\chordDiagram{G} & \chordDiagram{Am}
\end{tabular}
\end{center}

\begin{intro}
\meter{4}{4} | ^{C} | ^{F} | ^{C} | ^{F}
\end{intro}

\begin{verse}
| ^{C}I don't know you, | ^{F}but I want you | ^{C}all the more for that | ^{F} \\
| ^{C}Words fall through me and | ^{F}always fool me, | ^{C}and I can't react | ^{F}
\end{verse}

\begin{verse}
| ^{Am}Games that ^{G}never a|^{F}mount to ^{G}more than they're | ^{Am}meant will ^{G}play themselves | ^{F}out | ^{F}
\end{verse}

\begin{chorus}
| ^{C}Take this sinking | ^{F}boat and point it | ^{Am}home, we've still got | ^{F}time \\
| ^{C}Raise your hopeful | ^{F}voice, you had the | ^{Am}choice, you've made it | ^{F}now
\end{chorus}

\begin{verse}
| ^{C}Falling slowly, | ^{F}eyes that know me, | ^{C}and I can't ^{F}go back | ^{F} \\
| ^{C}Moods that take me | ^{F}and erase me, | ^{C}and I'm painted ^{F}black | ^{F}
\end{verse}

\begin{verse}
| ^{Am}You have ^{G}suffered e|^{F}nough and ^{G}warred with your|^{Am}self, it's ^{G}time that you | ^{F}won | ^{F}
\end{verse}

\begin{chorus}
| ^{C}Take this sinking | ^{F}boat and point it | ^{Am}home, we've still got | ^{F}time \\
| ^{C}Raise your hopeful | ^{F}voice, you had the | ^{Am}choice, you've made it | ^{F}now \\
| ^{C}Falling slowly, | ^{F}sing your melo|^{Am}dy, I'll sing it | ^{F}loud
\end{chorus}

\begin{outro}
| ^{C} | ^{F} | ^{Am} | ^{F} \\
| ^{C} | ^{F} | ^{C} | ^{F} | ^{C}
\end{outro}

\end{song}
\begin{song}{title={Knockin On Heavens Door}, music={Bob Dylan}}
\begin{center}
\begin{tabular}{rr}
\chordDiagram{C} & \chordDiagram{D} \\
\chordDiagram{G} & \chordDiagram{Am}
\end{tabular}
\end{center}

\begin{intro}
\meter{4}{4} | ^{G} ^{G} ^{D} ^{D} | ^{Am} | ^{G} ^{G}Oo ^{D}oo-oo | ^{C}oo \\
| ^{G} ^{G}Oo ^{D}oo-oo | ^{Am} | ^{G} ^{G}Oo ^{D}oo-^{D}oo | ^{C}oo
\end{intro}

\begin{verse}
| ^{G} ^{G}Mama take this ^{D}badge ^{D}off of me | ^{Am} \\
| ^{G} ^{G}I can't ^{D}use it ^{D}anymore | ^{C} \\
| ^{G} ^{G}It's getting ^{D}dark, too ^{D}dark to see | ^{Am} \\
| ^{G} ^{G}I feel I'm ^{D}knockin on ^{D}heaven's door | ^{C}
\end{verse}

\begin{chorus}
| ^{G} ^{G}Knock, knock, ^{D}knockin' on ^{D}heaven's | ^{Am}door \\
| ^{G} ^{G}Knock, knock, ^{D}knockin' on ^{D}heaven's | ^{C}door \\
| ^{G} ^{G}Knock, knock, ^{D}knockin' on ^{D}heaven's | ^{Am}door \\
| ^{G} ^{G}Knock, knock, ^{D}knockin' on ^{D}heaven's | ^{C}door
\end{chorus}

% pokračovat
\begin{verse}
^{G} Mama put my  ^{D}guns in the ground ^{Am} \\
^{G} I can't ^{D}shoot them anymore ^{C} \\
^{G} That long black ^{D}cloud is comin' down ^{Am} \\
^{G} I feel I'm ^{D}knockin' on heaven's door ^{C}
\end{verse}

\begin{chorus}
^{G} Knock, knock, ^{D}knockin' on heaven's ^{Am}door \\
^{G} Knock, knock, ^{D}knockin' on heaven's ^{C}door \\
^{G} Knock, knock, ^{D}knockin' on heaven's ^{Am}door \\
^{G} Knock, knock, ^{D}knockin' on heaven's ^{C}door
\end{chorus}

\begin{outro}
^{G} Oo ^{D}oo-oo ^{Am}oo ^{G}(fade)
\end{outro}
\end{song}

\end{document}